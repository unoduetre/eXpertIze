\documentclass[a4paper]{article}
\usepackage{ucs}
\usepackage[utf8x]{inputenc}
\usepackage[T1]{fontenc}
\usepackage{cancel}
\usepackage{graphicx}
\usepackage{dcolumn}
\usepackage{color}
\usepackage{enumerate}
\usepackage{url}
\usepackage[squaren]{SIunits}
\usepackage{icomma}
\usepackage{hyperref}
%\usepackage{url}
\usepackage{float}
\usepackage{indentfirst}
\usepackage[intlimits]{amsmath}
\usepackage{amsfonts}
\usepackage{verbatim}
\usepackage{tabulary}
\usepackage{longtable}
\usepackage{rotating}
\usepackage{formular}
\usepackage{marginnote}
\usepackage{listingsutf8}
\usepackage{ifpdf}
\usepackage{natbib}
\lstset{inputencoding=utf8/latin2,breaklines=true}
\newFRMfield{wyraz}{2cm}
\newif\ifShowAnswers
%\ShowAnswerstrue
\newcommand{\fillin}[2]{\ifShowAnswers{#1}\else{\marginnote{#2}\useFRMfield{wyraz}}\fi}
\newtheorem{definitionx}{Definicja}
\newenvironment{definition}{\begin{definitionx}\normalfont}{\end{definitionx}}
\ifpdf
\newcommand{\obraz}[2]{
\begin{figure}[H]
\centering
\label{fig:#1}
\includegraphics[width=12cm]{#1.pdf}
\caption{#2}
\end{figure}
}
\newcommand{\poziomyobraz}[1]{
\begin{sidewaysfigure}
\centering
\label{fig:#1}
\includegraphics[width=25cm]{#1.pdf}
%\caption{#2}
\end{sidewaysfigure}
}
\else
\newcommand{\obraz}[2]{
\begin{figure}[H]
\centering
\label{fig:#1}
\includegraphics[width=12cm]{#1.eps}
\caption{#2}
\end{figure}
}
\newcommand{\poziomyobraz}[1]{
\begin{sidewaysfigure}
\centering
\label{fig:#1}
\includegraphics[width=25cm]{#1.eps}
%\caption{#2}
\end{sidewaysfigure}
}
\fi
\newcommand{\rysunek}[4]{
\begin{figure}[H]
\centering
\label{fig:#1}
\begin{picture}#3
#4
\end{picture}
\caption{#2}
\end{figure}
}
\newcommand{\tabela}[4]{
\begin{table}[H]
\centering
\label{tab:#1}
\begin{tabular}[t]{#3}
#4
\end{tabular}
\caption{#2}
\end{table}
}

\newcommand{\tabelap}[4]{
\begin{table}[p]
\centering
\label{tab:#1}
\begin{tabular}[t]{#3}
#4
\end{tabular}
\caption{#2}
\end{table}
}

\newcommand{\tabelah}[4]{
\begin{table}[h]
\centering
\label{tab:#1}
\begin{tabular}[t]{#3}
#4
\end{tabular}
\caption{#2}
\end{table}
}

\newcommand{\tabelat}[4]{
\begin{table}[t]
\centering
\label{tab:#1}
\begin{tabular}[t]{#3}
#4
\end{tabular}
\caption{#2}
\end{table}
}

\newcommand{\tabelab}[4]{
\begin{table}[b]
\centering
\label{tab:#1}
\begin{tabular}[t]{#3}
#4
\end{tabular}
\caption{#2}
\end{table}
}

\newcommand{\ltabela}[3]{
\clearpage
\begin{longtable}[c]{#2}
\caption{#1}\\
#3
\end{longtable}

}

\title{Information and knowledge managment final project }
\author{Mateusz Grotek, Jože Kraner}
\date{}
\begin{document}
\maketitle
\begin{abstract}
The report describes ''eXpertIze'' program, which is a knowledge application software, for helping users to realize what they have in mind, fulfilling some constraints. It could be used both as a game, which allows to guess the name of an object a player has in mind by asking him/her some questions, and as a serious software module for discovering what a user has in mind when e.g. learning a language, and not remembering a word. The algorithm used is general. We focused on the game functionality.
\end{abstract}
\tableofcontents
\section{Introduction}
\section{Literature review}
\section{System architecture}
\section{Limitations}
\section{Conclusions and future work}
%The following references are just examples.
\begin{thebibliography}{000000000000000000000000000}
%\bibitem[Bernacki i in.(2005)]{BWG}Bernacki, M., Włodarczyk, P., Gołda, A. 2005. \textsl{Principles of training multi-layer neural network using backpropagation.} Katedra Elektroniki AGH, Kraków. \url{http://galaxy.agh.edu.pl/~vlsi/AI/backp_t_en/backprop.html}
%\bibitem[Frank i Asuncion(2010)]{FA}Frank, A., Asuncion, A. 2010. \textsl{{UCI} Machine Learning Repository: Iris Data Set.} University of California, Irvine, School of Information and Computer Science. \url{http://archive.ics.uci.edu/ml/datasets/Iris}
%\bibitem[Rojas(1996)]{RR}Rojas, R. 1996. \textsl{Neural Networks --- A Systematic Introduction.} Springer-Verlag, Berlin.
%\bibitem[Rutkowski(2005)]{LR}Rutkowski, L. 2005. \textsl{Metody i techniki sztucznej inteligencji.} PWN, Warszawa.
\end{thebibliography}
\end{document}

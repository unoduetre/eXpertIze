\documentclass[a4paper]{article}
\input{packages}
\title{Information and knowledge managment final project }
\author{Mateusz Grotek, Jože Kraner}
\date{}
\begin{document}
\maketitle
\begin{abstract}
The report describes ''eXpertIze'' program, which is a knowledge application software, for helping users to realize what they have in mind, fulfilling some constraints. It could be used both as a game, which allows to guess the name of an object a player has in mind by asking him/her some questions, and as a serious software module for discovering what a user has in mind when e.g. learning a language, and not remembering a word. The algorithm used is general. We focused on the game functionality.
\end{abstract}
\tableofcontents
\section{Introduction}
\section{Literature review}
\section{System architecture}
\section{Limitations}
\section{Conclusions and future work}
%The following references are just examples.
\begin{thebibliography}{000000000000000000000000000}
%\bibitem[Bernacki i in.(2005)]{BWG}Bernacki, M., Włodarczyk, P., Gołda, A. 2005. \textsl{Principles of training multi-layer neural network using backpropagation.} Katedra Elektroniki AGH, Kraków. \url{http://galaxy.agh.edu.pl/~vlsi/AI/backp_t_en/backprop.html}
%\bibitem[Frank i Asuncion(2010)]{FA}Frank, A., Asuncion, A. 2010. \textsl{{UCI} Machine Learning Repository: Iris Data Set.} University of California, Irvine, School of Information and Computer Science. \url{http://archive.ics.uci.edu/ml/datasets/Iris}
%\bibitem[Rojas(1996)]{RR}Rojas, R. 1996. \textsl{Neural Networks --- A Systematic Introduction.} Springer-Verlag, Berlin.
%\bibitem[Rutkowski(2005)]{LR}Rutkowski, L. 2005. \textsl{Metody i techniki sztucznej inteligencji.} PWN, Warszawa.
\end{thebibliography}
\end{document}

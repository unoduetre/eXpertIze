\documentclass[a4paper]{article}
\input{packages}
\title{Information and knowledge managment final project }
\author{Mateusz Grotek, Jože Kraner}
\date{}
\begin{document}
\maketitle
\begin{abstract}
The report describes ''eXpertIze'' program, which is a knowledge application software, for helping users to realize what they have in mind, fulfilling some constraints. It could be used both as a game, which allows to guess the name of an object a player has in mind by asking him/her some questions, and as a serious software module for discovering what a user has in mind when e.g. learning a language, and not remembering a word. The algorithm used is general. We focused on the game functionality.
\end{abstract}
\tableofcontents
\section{Introduction}
Let us consider the general knowledge discovery problem. The process of knowledge discovery was defined in \citet{Memon} as ''the development of new tacit or explicit knowledge from data and information or from the synthesis of prior knowledge''. Depending if an individual wants to gain explicit or tacit knowledge, possible strategies are combination of knowledge and socialization. 

Let us imagine the following situation: a user does not remember a word in a foreign language. He or she knows exactly what he or she wants to say. A knowledge discovery system could help him or her to find a proper word describing an object he or she has in mind by utilizing what he or she already knows about the thing. The same strategy could be used for the knowledge capturing process, which was defined in \citet{Memon} as ''the process of retrieving either explicit or tacit knowledge that resides within people, artifacts or organizational entities''. The difference between the two situations is the difference between the place the knowledge goes, it could be either a person or a knowledge base. In the knowledge discovery situation a person wants to gain some new knowledge. In the knowledge base situation an organization wants to transfer some knowledge from people's minds to organization's knowledge base. The source of the possibility of using the same algorithm lies in the following observation: in the presented situation a person is the source of knowledge. Nevertheless the end user of knowledge is different.

To be able to solve the problem we need to specify it further and to simplify it. The reasons for that are both time constraints and perceived difficulty of fully solving the problem. For this reason, we imagined the following scenario. A user has a material object in mind (but not \textbf{necessarily} a name of the object). By the least number of interactions with the user and by using a proper knowledge base we should be able to guess a name of the object. 
\section{Literature review}
%Mateusz
\section{System architecture}
%Database part - Joze. Mateusz - Engine
\section{Limitations}
%Joze

\section{Conclusions and future work}
% Joze

%The following references are just examples.
\begin{thebibliography}{10}
\bibitem[Memon(2011)]{Memon}Memon, N. 2011. \textsl{Information and Knowledge Management} --- lecture notes. University of Southern Denmark, Odense. \url{http://www.mip.sdu.dk/~memon/IKM_2011.htm}
\bibitem[Brin \& Page.(1998)]{Google} Brin, S. and Page, L. 1998. \textsl{The Anatomy of a Large-Scale Hypertextual Web Search Engine.} In: Seventh International World-Wide Web Conference, 1998. Brisbane, Australia.
%\bibitem[Bernacki i in.(2005)]{BWG}Bernacki, M., Włodarczyk, P., Gołda, A. 2005. \textsl{Principles of training multi-layer neural network using backpropagation.} Katedra Elektroniki AGH, Kraków. \url{http://galaxy.agh.edu.pl/~vlsi/AI/backp_t_en/backprop.html}
%\bibitem[Frank i Asuncion(2010)]{FA}Frank, A., Asuncion, A. 2010. \textsl{{UCI} Machine Learning Repository: Iris Data Set.} University of California, Irvine, School of Information and Computer Science. \url{http://archive.ics.uci.edu/ml/datasets/Iris}
%\bibitem[Rojas(1996)]{RR}Rojas, R. 1996. \textsl{Neural Networks --- A Systematic Introduction.} Springer-Verlag, Berlin.
%\bibitem[Rutkowski(2005)]{LR}Rutkowski, L. 2005. \textsl{Metody i techniki sztucznej inteligencji.} PWN, Warszawa.
\end{thebibliography}
\end{document}
